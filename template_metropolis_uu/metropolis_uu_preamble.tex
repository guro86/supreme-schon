\usetheme[progressbar=frametitle,numbering=fraction,block=fill]{metropolis}

%%% USEFUL PACKAGES
%\usepackage{showframe} % For debugging positioning
\usepackage{etex} % If too many packages
% Encoding and language
\usepackage[utf8]{inputenc}
\usepackage{babel}
\usepackage{amsmath, amssymb}
\usepackage{natbib}
%\usepackage{booktabs}
%\usepackage{algorithmic}
%\usepackage{algorithm}
%\usepackage{graphicx}
%\usepackage{caption}
%\usepackage{animate} % Animations
\usepackage{bm} % Bold math
\usepackage{bbm}
%\usepackage{url}
%\usepackage{pifont}
%\usepackage{ulem} % Used for strikeouts \sout
%\usepackage{stackengine}
%\usepackage{enumitem}
%\setlist[description]{leftmargin=\parindent,labelindent=\parindent}

\usepackage{algorithm,caption}

%%% COLORS
\usepackage{xcolor}
\definecolor{UURed}{HTML}{BF2D38}
% Colors derived from UURed, tetraedic at https://www.sessions.edu/color-calculator/
\definecolor{UUGreen}{HTML}{39BF2D}
\definecolor{DarkGreen}{HTML}{00B200}
\definecolor{UUBlue}{HTML}{2D43BF}
\definecolor{UUBrown}{HTML}{BF712D}
\definecolor{gold}{rgb}{.812,.710,.231}

%%% USEFUL MACROS
\setbeamercolor{background canvas}{bg=white}
% Text markup
%% UU Colors
%\setbeamercolor{alerted text}{fg=UURed}
%\newcommand{\blue}[1]{\textcolor{UUBlue}{#1}}
%\newcommand{\red}[1]{\textcolor{UURed}{#1}}
\newcommand{\blue}[1]{\textcolor{blue}{#1}}
\newcommand{\red}[1]{\textcolor{red}{#1}}
\newcommand{\grey}[1]{\textcolor{gray}{#1}}
\newcommand{\orange}[1]{\textcolor{mLightBrown}{#1}}
\newcommand\myheading[1]{\textbf{#1}}
\newcommand\myemph[1]{\underline{\emph{#1}}}
\newcommand\textexample[1]{\textit{\textbf{#1}}}
% Spacing
\newcommand\vws[1][1]{\vspace{#1\baselineskip}} % vertical white space
\newcommand\strt[1][1.5ex]{\rule[-.05\baselineskip]{0pt}{#1}} % strut
\newcommand\Hrule{\vspace{1ex} \hrule \vspace{1ex}} % Horisontal rule with some space after
% Misc
\newcommand\articleref[4]{\noindent\begin{minipage}[t]{0.04\textwidth}
		\vspace{0pt} 
		\pgfuseimage{beamericonarticle}
	\end{minipage}%
	\begin{minipage}[t]{0.96\textwidth}
		\vspace{0pt}
		#1. \textbf{#2.} \textit{#3}, #4.
	\end{minipage}}

%%% METROPOLIS THEME SPECIFIC
\makeatletter
\setlength{\metropolis@progressonsectionpage@linewidth}{1pt}
\makeatother
%\setbeamercolor{progress bar}{fg=UURed,bg=UURed!50}

%%% GRAPHICS
\usepackage{graphicx}
\graphicspath{{./Figures/}}

%%% TEXTPOS
\usepackage[absolute,overlay]{textpos} % option showboxes is useful in draft mode
\setlength{\TPHorizModule}{\paperwidth}
\setlength{\TPVertModule}{\paperheight}
\textblockorigin{0pt}{10mm} % start everything at top-left, below gray 

%% TIKZ
\usepackage{tikz}
\usetikzlibrary{arrows,calc} %,decorations.pathreplacing,backgrounds,positioning,fit,petri}

%%% BLOCKS AND BOXES
% Changing colors of blocks
%\setbeamercolor{block title alerted}{bg=UURed,fg=palette primary.fg}
%\setbeamercolor{block body alerted}{bg=UURed!15}
\setbeamercolor{block title alerted}{bg=mLightBrown,fg=palette primary.fg}
\setbeamercolor{block body alerted}{bg=mLightBrown!15}
%\setbeamercolor{block title example}{bg=UUGreen,fg=palette primary.fg}
%\setbeamercolor{block body example}{bg=UUGreen!10}
% \mybox is a rectangular box
\usepackage{boxedminipage}
\setlength\fboxrule{2pt}
\setlength\fboxsep{2\fboxsep}
\newcommand\mybox[3][\textwidth]{
  {\color{#2}
    \begin{boxedminipage}{#1}
      {\color{palette primary.bg} #3}
    \end{boxedminipage}}%
}   
\usepackage{tcolorbox}
\tcbset{arc=1mm,grow to left by=3mm,grow to right by=3mm,left=2mm}
%\newenvironment{redbox}{%
%	\begin{tcolorbox}[colback=UURed!15,colframe=UURed]}{%
%	\end{tcolorbox}}
%\newenvironment{greenbox}{%
%	\begin{tcolorbox}[colback=UUGreen!15,colframe=UUGreen]}{%
%	\end{tcolorbox}}
\newenvironment{redbox}{%
	\begin{tcolorbox}[colback=red!15,colframe=red]}{%
	\end{tcolorbox}}
\newenvironment{greenbox}{%
	\begin{tcolorbox}[colback=DarkGreen!15,colframe=DarkGreen]}{%
	\end{tcolorbox}}
\newenvironment{graybox}{%
	\begin{tcolorbox}[colback=mDarkTeal!5,colframe=mDarkTeal]}{%
	\end{tcolorbox}}
\newenvironment{orangebox}{%
\begin{tcolorbox}[colback=mLightBrown!15,colframe=mLightBrown]}{%
	\end{tcolorbox}}	


\newcommand{\backupbegin}{
	\newcounter{framenumberappendix}
	\setcounter{framenumberappendix}{\value{framenumber}}
}
\newcommand{\backupend}{
	\addtocounter{framenumberappendix}{-\value{framenumber}}
	\addtocounter{framenumber}{\value{framenumberappendix}} 
}

